%%%% Uniwersalny Szablon Infy WPPT v4.2 (02.02.2018) %%%%

% uklad dokumentu
	\documentclass{article}
	\usepackage{xparse}
	\usepackage[margin=2cm]{geometry}
	\usepackage{enumerate} 
	\frenchspacing
	\linespread{1.2}
 	\setlength{\parindent}{0pt}

% pakiety matematyczne
	\usepackage{amssymb}   
	\usepackage{amsthm}
	\usepackage{amsmath}
	\usepackage{amsfonts}
	\usepackage{tikz}

% jezyk polski
	\usepackage[polish]{babel}
	\usepackage[utf8]{inputenc}
	\usepackage{polski}
	\usepackage[T1]{fontenc}
 
% hiperlacza
	\usepackage{hyperref}
	\hypersetup{
		colorlinks,
		citecolor=black,
		filecolor=black,
		linkcolor=black,
		urlcolor=black
	}

% wstawianie zdjec
	\usepackage{graphicx} 

% deklaracja zadania
	\theoremstyle{definition}
	\newtheorem{zadanie}{Zadanie}[subsection]
	\renewcommand{\thezadanie}{\arabic{zadanie}}

% deklaracja metody
	\theoremstyle{remark}
	\newtheorem*{metoda}{Metoda}
	\theoremstyle{plain}
	
% deklaracja rozwiazania
	\theoremstyle{remark}
	\newtheorem*{rozwiazanie}{Rozwiązanie}
	\theoremstyle{plain}
	
% zestaw - mpis
	\newcommand{\EE}{\mathbb{E}}
	\newcommand{\PP}{\mathbb{P}}
	\newcommand{\Var}{\mathrm{Var}}
	\newcommand{\DD}{\mathbb{D}}
	\newcommand{\red}[1]{\textcolor{red}{#1}}

% zestaw - programowanie
	\usepackage{listings} 
	\usepackage{minibox}
	% \usepackage{minted}
	% \usemintedstyle{borland}

% zestaw - akiso
	%\usepackage{karnaugh-map} %%% nie ma w bazie overleaf
	\usepackage{karnaughmap}
	\usepackage{circuitikz}
	
	\usepackage{tikz}
    \usetikzlibrary{automata,positioning}
    
% zestaw - dyskretna
	\newcommand{\stirf}[2]{\genfrac{[}{]}{0pt}{}{#1}{#2}}
	\newcommand{\stirs}[2]{\genfrac{\{}{\}}{0pt}{}{#1}{#2}}
	
% zestaw - jftt
	\newcommand{\oor}{\mathop{|}}
	\DeclareMathOperator{\Lead}{Leading}
	\DeclareMathOperator{\Trail}{Trailing}
	\DeclareMathOperator{\fst}{First}
	\DeclareMathOperator{\fol}{Follow}
	
\usepackage{multicol}
\usepackage{multirow}
\usepackage{float}
\title{Metody Optymalizacji - Lista 1}
\author{Janusz Witkowski 254663}
\date{2 kwietnia 2023}

\begin{document}

\maketitle

\section{Zadanie 1}
\subsection{Model}
\subsubsection{Zmienne decyzyjne}
Użyta została tylko jedna zmienna decyzyjna $\textbf{\textit{X}}$, która jest wektorem spełniającym warunek $\textbf{\textit{X}} \geq \textbf{0}$.
Teoretycznym rozwiązaniem problemu będącego zagadnieniem tego zadania jest wektor $\textbf{\textit{Y}} \textbf{ = 1} $.
Będziemy mogli później porównać wynikowy wektor $\textbf{\textit{X}}$ z $\textbf{\textit{Y}}$, aby przekonać się o precyzji GNU MathProg.

\subsubsection{Ograniczenia}
Za ograniczenia przyjmujemy równania liniowe podane w treści zadania:
\[A\textbf{\textit{x}}=\textbf{\textit{b}}\]

przy warunkach

\[a_{ij} = \frac{1}{i+j-1}, i,j \in 1,...,n\]
\[c_i = b_i = \sum_{j=1}^n{\frac{1}{i+j-1}}, i \in 1,...,n\]

Macierz $A$ jest tzw. macierzą Hilberta, powodującą złe uwarunkowanie zagadnienia.

\subsubsection{Funkcja celu}
Funkcja celu ma postać podaną w treści polecenia 

\[ min {\textit{{\textbf{ c}}}}^T \textbf{\textit{x}} \]

\subsection{Wyniki}
Rozważmy rozwiązania modelu dla $n \in 1,\ldots,12$.
\begin{table}[H]
    \centering
    \begin{tabular}{|l|l|}\hline
    n & Błąd względny \\\hline
    1 & $0$\\\hline
    2 & $1.05325\cdot 10^{-15}$\\\hline
    3 & $3.67158\cdot 10^{-15}$\\\hline
    4 & $3.2701639\cdot 10^{-13}$\\\hline
    5 & $3.35139917\cdot 10^{-12}$\\\hline
    6 & $6.833357907\cdot 10^{-11}$\\\hline
    7 & $1.678685421923\cdot 10^{-8}$\\\hline
    8 & 0.51405897217726825854\\\hline
    9 & 0.68291133808772241931\\\hline
    10 & 0.99038757480308581282\\\hline
    11 & 0.95219663921467334600\\\hline
    12 & 1.29250739578124163032\\\hline
    \end{tabular}
\end{table}


Zauważmy przeskok błędu z $n = 7$ na $n = 8$. Od tego momentu rozwiązania problemu
nie da się już wyznaczyć z precyzją do dwóch cyfr. Wzrost błędu względnego jest 
oczywiście spowodowany złym uwarunkowaniem zadania macierzy Hilberta.
        
\section{Zadanie 2}
\subsection{Model}
\subsubsection{Zmienne decyzyjne}
Chcemy w pewien sposób zamodelować rozkład jazdy dźwigów różnego typu z jednych miast do innych.
Na razie przyjmijmy, że dźwigów nie da się podzielić, żebyśmy obracali się w programowaniu całkowitoliczbowym.
Zdefiniujmy następującą zmienną decyzyjną:

\[ x \in \mathbb{Z}^{n\times n\times 2}, x \geq 0 \]

Przez $x_{abt}$ będziemy rozumieć ile dźwigów typu $t$ należy przetransportować z miasta $a$ do miasta $b$.

\subsubsection{Parametry}
Wprowadźmy pewne parametry, ściśle związane z problemem.

\[ d \in \mathbb{Z}^{2\times n}, d \geq 0 \]
\[ r \in \mathbb{Z}^{2\times n}, r \geq 0 \]
\[ z \in \{0,1\}^{2\times 2} \]
\[ l \in \mathbb{R}^{n\times n}, l \geq 0 \]
\[ k \in \mathbb{R}^{2}, k \geq 0 \]

Niech $d_{ta}$ oznacza ile dźwigów typu $t$ brakuje w mieście $a$.

Niech $r_{ta}$ oznacza nadmiar dźwigów typu $t$ w mieście $a$.

Niech $z_{ts}$ oznacza czy dźwig typu $t$ może zastąpić dźwig typu $s$.

Niech $l_{ab}$ oznacza dystans w kilometrach między miastem $a$ a miastem $b$.
Dystanse te wyciągnięto za pomocą API do GoogleMaps.

Niech $k_t$ oznacza koszt transportu dźwigu typu $t$ za każdy przebyty kilometr.

\subsubsection{Ograniczenia}
Aby zagwarantować dopuszczalność potencjalnych rozkładów transportu, zdefiniujmy trzy ograniczenia:
\begin{enumerate}
    \item \textbf{Nasyć popyt na różne typy} - każde miasto powinno otrzymać takie dźwigi aby mogło funkcjonować (dźwigi typu II mogą zastąpić dźwigi typu I)
    % \[ (\forall a \in \{1,\dots,n\}, t \in \{I,II\}) \left(\sum_{b=1}^{n-1}{x_{abt} \leq d_{ak}}\right) \]
    \[ (\forall a \in \{1,\dots,n\}, t \in \{I,II\}) \left(\sum_{b=1}^{n}{\sum_{s \in \{I,II\}}{x_{abs} \cdot z_{st}}} \geq d_{ta} \right) \]

    \item \textbf{Z pustego nawet Salomon nie naleje} - miasto nie może oddać więcej dźwigów niż samo posiada
    \[ (\forall a \in \{1,\dots,n\}, t \in \{I,II\}) \left(\sum_{b=1}^{n}{x_{abt}} \leq r_{ta} \right) \]

    \item \textbf{Nie zgub żadnego dźwigu po drodze} - potrzebujące miasta powinny dostać łącznie tyle dźwigów by zaspokoić swoje niedobory
    \[ (\forall a \in \{1,\dots,n\}) \left(\sum_{b=1}^{n}{\sum_{t \in \{I,II\}}{x_{abs}}} \geq \sum_{s \in \{I, II\}}{d_{sa}} \right) \]
\end{enumerate}

\subsubsection{Funkcja celu}
Chcemy zminimalizować koszty transportu dźwigów między miastami. Zaproponujmy następującą funkcję celu:

\[ min {\sum_{t \in \{I,II\}}{\sum_{a=1}^{n}{\sum_{b=1}^{n}{l_{ab} \cdot x_{abt} \cdot k_t}}}} \]

\subsection{Wyniki}
Koszt znalezionego rozwiązania optymalnego wyniósł 1424.4 (zakładając, że za przewóz dźwigu typu I płaci się 1).
Poniższa tabelka ilustruje wynikowy rozkład jazdy. Lewa kolumna wskazuje na eksportera, górny wiersz na importera,
natomiast każda komórka na to ile przewozi się dźwigów typu I | typu II.

\begin{table}[H]
    \centering
    \begin{tabular}{|l|c|c|c|c|c|c|c|} \hline
     & Opole & Brzeg & Nysa & Prudnik & Strzelce & Koźle & Racibórz\\\hline
    Opole & -|- & 7|- & -|- & -|- & -|- & -|- & -|-\\\hline
    Brzeg & -|- & -|1 & -|- & -|- & -|- & -|- & -|-\\\hline
    Nysa & -|2 & 2|- & -|- & 1|- & -|- & 3|- & -|-\\\hline
    Prudnik & -|- & -|- & -|- & -|3 & -|4 & -|2 & -|1\\\hline
    Strzelce & -|- & -|- & -|- & -|- & -|- & 5|- & -|-\\\hline
    Koźle & -|- & -|- & -|- & -|- & -|- & -|- & -|-\\\hline
    Racibórz & -|- & -|- & -|- & -|- & -|- & -|- & -|-\\\hline
    \end{tabular}
\end{table}

Niektóre miasta przesyłają same sobie dźwigi typu II, co na pierwszy rzut oka może wydać się dziwne.
Można to potraktować jako sygnał że dźwigi te zastępują dźwigi typu I w obrębie danego miasta.
Taka tranzakcja niczego nie zmienia w kwestii funkcji celu, ponieważ założyliśmy że odległość z miasta do tego samego miasta wynosi 0.

Założenie całkowitoliczbowości nic nie daje w tym przypadku - wyniki są identyczne w przypadku zamiany na zmienne ciągłe.

\section{Zadanie 3}
\subsection{Model}
\subsubsection{Zmienne decyzyjne}
Model do tego zadania będzie trochę bardziej skomplikowany.
Będziemy się tu posługiwać zbiorami: typów ropy $Ropy$, typów paliw wynikowych $Paliwa$ oraz typów produktów $Produkty$ i $ProduktyKrakowane$.
Będziemy się też posługiwać zbiorem procesów (destylacja, krakowanie) o nazwie $Procesy$.

Zdefiniujmy 3 zmienne decyzyjne.

\[ x \in \mathbb{R}^{Ropy}, x \geq 0 \]
\[ rdest \in \mathbb{R}^{Ropy\times {0,1}}, rdest \geq 0 \]
\[ rolej \in \mathbb{R}^{Ropy\times Paliwa}, rolej \geq 0 \]

$x_{ropa}$ to ilość potrzebnej do kupienia ropy typu $ropa$ [w tonach].

$rdest_{ropa, b}$ oznacza ile destylatu z ropy typu $ropa$ idzie na krakowanie ($b=1$), a ile do dalszej produkcji ($b=0$).

$rolej_{ropa, paliwo}$ wskazuje ile oleju przydzielimy do danego paliwa $paliwo$ (tak naprawdę liczą się tylko Domowe i Ciężkie).

\subsubsection{Parametry}
Zdefiniujmy takie parametry, ściśle zależne od problemu.

\[ wdes \in \mathbb{R}^{Produkty\times Ropy}, 0 \leq wdes \leq 1 \]
\[ wkra \in \mathbb{R}^{ProduktyKrakowane}, 0 \leq wdes \leq 1 \]
\[ kropy \in \mathbb{R}^{Ropy}, kropy \geq 0 \]
\[ kproc \in \mathbb{R}^{Procesy}, kproc \geq 0 \]
\[ w \in \mathbb{R}^{Paliwa}, w \geq 0 \]
\[ szaw \in \mathbb{R}^{Procesy\times Ropy}, 0 \leq szaw \leq 1 \]
\[ smax \in \mathbb{R}, 0 \leq smax \leq 1 \]

gdzie
\begin{itemize}
    \item $wdes_{produkt, ropa}$ - wydajność destylacji,
    \item $wkra_{produkt}$ - wydajność krakowania,
    \item $kropy_{ropa}$ - koszt kopy na tonę,
    \item $kproc_{proces}$ - koszt procesu na tonę przetworzonego składnika,
    \item $w_{paliwo}$ - wymagana produkcja danego paliwa,
    \item $szaw_{proces, ropa}$ - zawartość siarki w oleju pochodzenia danej ropy z danego procesu,
    \item $smax$ - maksymalna dozwolona zawartość siarki w paliwie domowym.
\end{itemize}

\subsubsection{Ograniczenia}


\subsubsection{Funkcja celu}
Chcemy zminimalizować wszystkie koszta jakie rafineria może ponieść, czyli zarówno te od kupna ropy,
jak i te z procesów. Zaproponujmy następującą funkcję celu.

\[ \sum_{r \in Ropy}{(kropy_r + kproc_{Destylacja})\cdot x_r + rdest_{r,1}\cdot kproc_{Krakowanie}} \]


\subsection{Wyniki}
Całkowite koszta poniesione przez rafinerię, według rozwiązania z solvera, nie powinny przekroczyć \$1345943600.867679 dolarów.
Poniżej przedstawione są poszczególne zakupione ropy:
\begin{table}[H]
    \centering
        \begin{tabular}{|c|c|}
            \hline
            Ropa & Ilość [t]           \\ \hline
            B1   & 1026030.368764 \\ \hline
            B2   & 0.0          \\ \hline
        \end{tabular}
\end{table}

Najwyraźniej kupowanie jakichkolwiek ilości ropy B2 nie jest postrzegane jako optymalne - zupełnie wystarczającą jest ropa B1.

\section{Zadanie 4}
\subsection{Model}
\subsubsection{Zmienne decyzyjne}
W modelu wykorzystano następujące zmienne decyzyjne:

\[ x \in \{0,1\}^{n\times m} \]
\[ t \in \{0,1\}^{\{1,\ldots,5\}\times \{1,\ldots,48\}} \]
\[ s \in \{0,1\}^{\{1,2,3\}} \]

$x_{gc}$ oznacza zapis Studenta na kurs $c$ do grupy $g$. 
$t_{dh}$ indykuje czy Student jest zajęty w dniu $d$ o godzinie $h/2$.
$s_k$ mówi czy Student zapisał się na zajęcia sportowe $k$.

\subsubsection{Parametry}
W modelu wykorzystano następujące parametry, ściśle powiązane z problemem:

\[ r \in \mathbb{Z}^{n\times m}, 0 \leq r \leq 10 \]
\[ w \in \mathbb{Z}^{n\times m}, 1 \leq w \leq 5 \]
\[ b \in \mathbb{Z}^{n\times m}, 1 \leq b \leq 48 \]
\[ e \in \mathbb{Z}^{n\times m}, 1 \leq e \leq 48 \]
\[ sw \in \mathbb{Z}^{[3]}, 1 \leq sw \leq 5 \]
\[ sb \in \mathbb{Z}^{[3]}, 1 \leq sb \leq 48 \]
\[ se \in \mathbb{Z}^{[3]}, 1 \leq se \leq 48 \]

$r_{gc}$ jest oceną (priorytetem) grupy, ustawioną przez Studenta. 
$w_{gc}$ mówi w którym dniu tygodnia odbywają się zajęcia grupy $g$ kursu $c$.
$b_{gc}$ oznacza o której pół-godzinie zajęcia się zaczynają.
$e_{gc}$ oznacza o której pół-godzinie zajęcia się kończą.
Analogicznie dla $sw$, $sb$, $se$ dla zajęć sportowych.

\subsubsection{Ograniczenia}
Musimy rozpisać kilka ograniczeń do określenia modelu, zarówno dla podstawowego wariantu zadania, jak i rozszerzonego.
\begin{enumerate}
    \item \textbf{Realizacja programu studiów} - Student musi zapisać się do dokładnie jednej grupy zajęciowej dla każdego kursu
    \[ (\forall c \in [m]) \left(\sum_{g \in [n]}{x_{gc}}\right) = 1 \]

    \item \textbf{Sport to zdrowie} - Student chce zapisać się na co najmniej jedną z trzech możliwych grup z zajęć sportowych
    \[ \sum_{i \in [3]}{s_{i}} \geq 1 \]

    \item \textbf{Nie żyjemy w kwantowej superpozycji} - zajęcia nie mogą się na siebie nakładać (zapiszemy tu od razu terminy )
    {\footnotesize
    \[ (\forall d \in [5], h \in [24*2]) t_{dh} = \left(\sum_{g \in [n]}{\sum_{c \in [m]}{x_{gc}\cdot (\textbf{if } d_{gc} = d \textbf{ then } 1 \textbf{ else } 0)\cdot (\textbf{if } h < b_{gc} \textbf{ then } 0 \textbf{ else } (\textbf{if } h \geq e_{gc} \textbf{ then } 0 \textbf{ else } 1))}} \right) + \]
    \[ \left(\sum_{i \in [3]}{(\textbf{if } sw_i \neq d \textbf{ then } 0 \textbf{ else } (\textbf{if } h < sb_i \textbf{ then } 0 \textbf{ else } (\textbf{if } h \geq se_i \textbf{ then } 0 \textbf{ else } 1)))} \right) \]
    }
    % \[ (\forall a \in \{1,\dots,n\}, t \in \{I,II\}) \left(\sum_{b=1}^{n}{x_{abt}} \leq r_{ta} \right) \]

    \item \textbf{Nie przemęczaj się} - maksymalnie 4 godziny zajęć obowiązkowych w dniu
    \[ (\forall d \in [5]) \left(\sum_{g \in [n]}{\sum_{c \in [m]}{x_{gc}\cdot (\textbf{if } d_{gc} = d \textbf{ then } b_{gc} - e_{gc} \textbf{ else } 0)}} \right) \]
    % \[ (\forall a \in \{1,\dots,n\}) \left(\sum_{b=1}^{n}{\sum_{t \in \{I,II\}}{x_{abs}}} \geq \sum_{s \in \{I, II\}}{d_{sa}} \right) \]

    \item \textbf{Przez żołądek do mózgu} - każdego dnia znajdź co najmniej 1 godzinę przerwy między 12:00 a 14:00 (tj. nie zajmuj w tym okresie więcej niż 1 godziny zajęciami)
    \[ (\forall d \in [5]) \left(\sum_{h \in \{12\cdot 2,\ldots,14\cdot 2 - 1\}}{t_{dh}}\right) <= 1\cdot 2 \]

    \item \textit{[Rozszerzenie]} \textbf{Napięty grafik tygodnia} - środy i piątki mają być wolne od zajęć obowiązkowych
    \[ (\forall d \in \{3,5\}) \left(\sum_{g \in [n]}{\sum_{c \in [m]}{x_{gc}\cdot (\textbf{if } w_{gc} = d \textbf{ then } 1 \textbf{ else } 0)}} \right) = 0 \]

    \item \textit{[Rozszerzenie]} \textbf{Ponadprzeciętna satysfakcja} - wybieraj kursy o ocenie równej co najmniej 5
    \[ \left(\sum_{g \in [n]}{\sum_{c \in [m]}{x_{gc}\cdot (\textbf{if } r_{gc} < 5 \textbf{ then } 1 \textbf{ else } 0)}} \right) = 0 \]
\end{enumerate}

\subsubsection{Funkcja celu}
Student chce zmaksymalizować sumę punktów priorytetowych z każdych zajęć na które się zapisze.

\[ max {\sum_{g \in [n]}{\sum_{c \in [m]}{x_{gc}\cdot r_{gc}}}} \]

\subsection{Wyniki}
Znaleziony podstawowy grafik ma wartość funkcji celu równą 37 i wygląda następująco:
\begin{table}[H]
	\centering
	\begin{tabular}{|c|c|} \hline
		Kurs & Grupa \\\hline
        Algebra & III \\\hline
        Analiza & II \\\hline
        Fizyka & IV \\\hline
        Minerały & I \\\hline
        Organiczna & II \\\hline
	\end{tabular}
\end{table}

Możemy zobaczyć jak wygląda ten grafik godzinowo:
\begin{table}[H]
	\centering
	\begin{tabular}{c|c|c|c|c|c}
		& Pn. & Wt. & Śr. & Cz. & Pt.\\\hline
		8:00 & x & & & & \\\hline
		8:30 & x & & & & \\\hline
		9:00 & x & & & & \\\hline
		9:30 & x & & & & \\\hline
		10:00 & & x & x & & \\\hline
		10:30 & x & x & x & & \\\hline
		11:00 & x & x & x & & \\\hline
		11:30 & x & x & x & & \\\hline
		12:00 & & & & & \\\hline
		12:30 & & & & & \\\hline
		13:00 & & & & & \\\hline
		13:30 & & & & & \\\hline
		14:00 & & & & & \\\hline
		14:30 & & & & & \\\hline
		15:00 & & & & & \\\hline
		15:30 & & & & & \\\hline
		16:00 & & & & & \\\hline
		16:30 & & & & & \\\hline
		17:00 & & & & x & \\\hline
		17:30 & & & & x & \\\hline
		18:00 & & & & x & \\\hline
		18:30 & & & & x & \\\hline
		19:00 & & & & x & \\\hline
		19:30 & & & & x & \\\hline
        20:00 & & & & & \\\hline
	\end{tabular}
\end{table}

Widać, że Student nie ma więcej niż 4 godzin ćwiczeń dziennie, ma możliwe przerwy na lunch, oraz że może uprawiać sport w co najmniej jednej grupie (tj. Pn 13:00-15:00).

Po wczytaniu modelu dla wariantu rozszerzonego, udało się odnaleźć takie oto rozwiązanie z wartością funkcji celu równą 28:
\begin{table}[H]
	\centering
	\begin{tabular}{|c|c|} \hline
		Kurs & Grupa \\\hline
        Algebra & I \\\hline
        Analiza & IV \\\hline
        Fizyka & II \\\hline
        Minerały & III \\\hline
        Organiczna & II \\\hline
	\end{tabular}
\end{table}

\begin{table}[H]
	\centering
	\begin{tabular}{c|c|c|c|c|c}
		& Pn. & Wt. & Śr. & Cz. & Pt.\\\hline
		8:00 & & & & x & \\\hline
		8:30 & & & & x & \\\hline
		9:00 & & & & x & \\\hline
		9:30 & & & & x & \\\hline
		10:00 & & x & & & \\\hline
		10:30 & x & x & & & \\\hline
		11:00 & x & x & & & \\\hline
		11:30 & x & x & & & \\\hline
		12:00 & & x & & & \\\hline
		12:30 & & x & & & \\\hline
		13:00 & x & & & x & \\\hline
		13:30 & x & & & x & \\\hline
		14:00 & x & & & x & \\\hline
		14:30 & x & & & x & \\\hline
		15:00 & & & & & \\\hline
		15:30 & & & & & \\\hline
		16:00 & & & & & \\\hline
		16:30 & & & & & \\\hline
		17:00 & & & & & \\\hline
		17:30 & & & & & \\\hline
		18:00 & & & & & \\\hline
		18:30 & & & & & \\\hline
		19:00 & & & & & \\\hline
		19:30 & & & & & \\\hline
        20:00 & & & & & \\\hline
	\end{tabular}
\end{table}

Tutaj Student również ma swobodę w lunchu i w uprawianiu sportu (np. grupa Śr 11:00-13:00).

\end{document}
